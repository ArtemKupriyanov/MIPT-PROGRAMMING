\documentclass[a4paper,12pt]{paper} % добавить leqno в [] для нумерации слева

%%% Работа с русским языком
\usepackage{cmap}					% поиск в PDF
\usepackage[T2A]{fontenc}			% кодировка
\usepackage[utf8]{inputenc}			% кодировка исходного текста
\usepackage[english,russian]{babel}	% локализация и переносы
\usepackage{indentfirst}

%%% Дополнительная работа с математикой
\usepackage{amsmath,amsfonts,amssymb,amsthm,mathtools} % AMS
\usepackage{icomma} % "Умная" запятая: $0,2$ --- число, $0, 2$ --- перечисление

%% Номера формул
\mathtoolsset{showonlyrefs=true} % Показывать номера только у тех формул, на которые есть \eqref{} в тексте.

%% Шрифты
\usepackage{euscript}	 % Шрифт Евклид
\usepackage{mathrsfs} % Красивый матшрифт

%% Свои команды
\DeclareMathOperator{\sgn}{\mathop{sgn}}

%% Перенос знаков в формулах (по Львовскому)
\newcommand*{\hm}[1]{#1\nobreak\discretionary{}
	{\hbox{$\mathsurround=0pt #1$}}{}}

\makeatletter
\def\@seccntformat#1{%
  \expandafter\ifx\csname c@#1\endcsname\c@section\else
  \csname the#1\endcsname\quad
  \fi}
\makeatother

%%% Заголовок
%%% Начало документа
\begin{document}
\section{Куприянов Артем. \\ Идеальное паросочетание \\ Тип задачи: теоретическая.
Баллы: 2.} 

Пусть дано: $G = (V, E)$ -- двудольный граф с долями $L$ и $R$. Определим для любого $X$, подмножества $V$, множество $N(X)$ соседей, т.е. вершин, соединенных с $X$ ребром. Докажите лемму Холла. 

\textbf{Лемма Холла: } Паросочетание размера $\vert L \vert$ существует тогда и только тогда, когда для любого $A$, подмножества $L$, верно $\vert A \vert \leq \vert N(A) \vert$.

\textbf{Доказательство леммы Холла:} \\
$\vartriangleright$  
$\Rightarrow$ \\ Пусть существует паросочетание размера $\vert L \vert$. Тогда очевидно, что для любого $A \subset L$ выполнено $\vert A \vert \leq \vert N(A) \vert$. Ведь, у любого подмножества вершин есть по крайней мере столько же соседей по паросочетанию. 
\\ $\Leftarrow$ \textbf{(По индукции) } \\
Есть изначально пустое паросочетание $P$. Будем добавлять на каждом шаге одно ребро и доказывать, что мы можем это сделать, если $\vert \{ \ x \  \vert \  x \in P \wedge x \in L \  \} \vert < \vert L \vert$.

\textbf{База: } \\ Возьмем любую вершину из $L$. Она соединена хотя бы с одной вершиной из $R$ (из условия, что для любого $A \subset L$ выполнено $\vert A \vert \leq \vert N(A) \vert$).

\textbf{Шаг: } \\
Пусть после  $k < \vert L \vert$ шагов построено какое-то паросочетание $P$. Докажем, что в это паросочетание можно добавить вершину  $v \in L$, еще не насыщенную этим паросочетанием.
 
Рассмотрим множество вершин $H_v$ -- множество вершин, достижимых из $v$, если из $L$ в $R$ можно ходить по любым ребрам нашего двудольного графа, а из $R$ в $L$ только по ребрам, принадлежащим нашему паросочетанию $P$.

Покажем, что в $H_v$  есть вершина $u \in R$, не насыщенная текущим паросочетанием $P$. 

От противного: пусть, нет такой вершины. Тогда рассмотрим вершины $H_L \subset H_v$, лежащие в $L$. Для них будет выполнено условие $\vert H_L \vert > \vert N(H_L) \vert$ (следует из построения $H$  -- из $L$ в $R$ можно ходить по любым ребрам нашего двудольного графа и понятия соседей вершины. $\Rightarrow$ пришли к против противоречию $\Rightarrow$ такая вершина существует.

Тогда существует путь из  $v$ в $u$, который будет удлиняющим для текущего паросочетания $P$. Значит, мы нашли увеличивающуюся цепь $v \mapsto \cdots \mapsto u$ $\Rightarrow$ (По теореме о максимальном паросочетании и дополняющих цепях), паросочетание $P$ не будет являться максимальным и мы можем увеличить текущее паросочетание вдоль найденной цепи. Получим бОльшее паросочетание. Шаг индукции доказан.

$\Rightarrow$ Доказав такую индукцию, получаем в конце процесса искомое паросочетание размера $\vert L  \vert$.

$\blacktriangleleft$   


\end{document}