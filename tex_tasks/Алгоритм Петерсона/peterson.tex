\documentclass[a4paper,12pt]{paper} % добавить leqno в [] для нумерации слева

%%% Работа с русским языком
\usepackage{cmap}					% поиск в PDF
\usepackage[T2A]{fontenc}			% кодировка
\usepackage[utf8]{inputenc}			% кодировка исходного текста
\usepackage[english,russian]{babel}	% локализация и переносы
\usepackage{indentfirst}

%%% Дополнительная работа с математикой
\usepackage{amsmath,amsfonts,amssymb,amsthm,mathtools} % AMS
\usepackage{icomma} % "Умная" запятая: $0,2$ --- число, $0, 2$ --- перечисление

%% Номера формул
\mathtoolsset{showonlyrefs=true} % Показывать номера только у тех формул, на которые есть \eqref{} в тексте.

%% Шрифты
\usepackage{euscript}	 % Шрифт Евклид
\usepackage{mathrsfs} % Красивый матшрифт

%% Свои команды
\DeclareMathOperator{\sgn}{\mathop{sgn}}

%% Перенос знаков в формулах (по Львовскому)
\newcommand*{\hm}[1]{#1\nobreak\discretionary{}
	{\hbox{$\mathsurround=0pt #1$}}{}}

\makeatletter
\def\@seccntformat#1{%
  \expandafter\ifx\csname c@#1\endcsname\c@section\else
  \csname the#1\endcsname\quad
  \fi}
\makeatother

%%% Заголовок

%%% Начало документа
\begin{document}
\section{Куприянов Артем. \\ Алгоритм Петерсона - 2 балла} 
\textbf{Ответ: } Нет, такая "модификация" \ алгоритма Петерсона для двух потоков не будет гарантировать взаzимное исключение.


\textbf{Решение: } \\
Приведем пример последовательного исполнения в модели чередования инструкций на одном процессоре, в котором вызов mtx.lock() из двух потоков приводит к нарушению взаимного исключения.:
Пусть есть 2 потока -  $A$ и $B$. Делаем mtx.lock(). \\
1) \textbf{Поток A станет victum.} victim=A, want[A] = false, want[B] = false. \\
2) \textbf{Поток B станет victum.} victim=B, want[A] = false, want[B] = false \\
3) \textbf{Поток B возведет свой флажок.} victim=B, want[A] = false, want[B] = true\\
4) \textbf{Поток B входит в критическую секцию.} \\ Покажем это: на момент входа условие (want[A].load() \&\& victum.load() == B) -- это false \\
5) \textbf{Поток A возведет свой флажок.} victim=B, want[A] = true, want[B] = true\\
6) \textbf{Поток A тоже войдет в критическую секцию.} \\ Покажем это: на момент входа условие (want[B].load() \&\& victum.load() == A) -- это false. Поэтому и поток A войдем в критическую секцию. \\
А значит, нарушится взаимное исключение. 

\textbf{ЧТД}

\end{document}