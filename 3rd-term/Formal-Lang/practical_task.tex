\documentclass[a4paper,12pt]{paper} % добавить leqno в [] для нумерации слева

%%% Работа с русским языком
\usepackage{cmap}					% поиск в PDF
\usepackage[T2A]{fontenc}			% кодировка
\usepackage[utf8]{inputenc}			% кодировка исходного текста
\usepackage[english,russian]{babel}	% локализация и переносы
\usepackage{indentfirst} % Красная строка

%%% Дополнительная работа с математикой
\usepackage{amsmath,amsfonts,amssymb,amsthm,mathtools} % AMS
\usepackage{icomma} % "Умная" запятая: $0,2$ --- число, $0, 2$ --- перечисление

%% Номера формул
\mathtoolsset{showonlyrefs=true} % Показывать номера только у тех формул, на которые есть \eqref{} в тексте.

%% Шрифты
\usepackage{euscript}	 % Шрифт Евклид
\usepackage{mathrsfs} % Красивый матшрифт

%% Свои команды
\DeclareMathOperator{\sgn}{\mathop{sgn}}

%% Перенос знаков в формулах (по Львовскому)
\newcommand*{\hm}[1]{#1\nobreak\discretionary{}
	{\hbox{$\mathsurround=0pt #1$}}{}}

\makeatletter
\def\@seccntformat#1{%
  \expandafter\ifx\csname c@#1\endcsname\c@section\else
  \csname the#1\endcsname\quad
  \fi}
\makeatother

%%% Заголовок
%%% Начало документа
\begin{document}
\section{Куприянов Артем.} 
\textbf{Задача 7. } Даны $\alpha$, буква $x$ и натуральное число $k$. Вывести длину кратчайшего слова из языка $L$, содержащее подслово длины $x^{k}$.
\\ 

\textbf{Алгоритм:} \\

1) Строим НКА по данному регулярному выражению. \\

2) Запускаем bfs из стартовой вершины в поисках пути (цикла) длины $k$, состоящего из  $x$. Как это делаем? 


\ \ 2.а) если в начале встречаем ребро $x$, то кладем в массив тройку (Вершина, из которой начинается путь, состоящий только из  $x$ вершина, из которой заканчивается этот путь, длина пути). Если длина пути, больше $k$, то запоминаем этот путь, как нужный далее для работы алгоритма. На шаге bfs: 

\ \ 2.б) если мы пришли в какую-то либо вершину $b$ по ребру $x$ из  $a$, то ищем в нашем массиве такую тройку, что она заканчивается вершиной $a$ и добавляем в наш массив новую тройку с обновленной длиной и конечной вершиной.

\ \ 2.в) если мы пришли в какую-то либо вершину $b$ по ребру не $x$ из  $a$, то, делаем пункт а) \\

3) Таким образом мы найдем все пути(циклы) длины   $k$. Запустим Алгоритм Флойда-Уоршелла для поиска кратчайших путей в нашем автомате. 

Для каждого полученного нами цикла длины $к$ найдем: min(Расстояние от стартовой вершины до начала цикла + длина цикла + расстояние от конечной вершины до завершающих состояний). Минимум этой величины по всем циклам будет ответом на нашу задачу.


\end{document}