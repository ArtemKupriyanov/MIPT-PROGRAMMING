\documentclass[a4paper,12pt]{article} % добавить leqno в [] для нумерации слева

%%% Работа с русским языком
\usepackage{cmap}					% поиск в PDF
\usepackage[T2A]{fontenc}			% кодировка
\usepackage[utf8]{inputenc}			% кодировка исходного текста
\usepackage[english,russian]{babel}	% локализация и переносы
\usepackage{indentfirst}
\usepackage[left=2.5cm,right=2.5cm,
    top=2cm,bottom=2cm,bindingoffset=0cm]{geometry}
   
\usepackage{graphicx}
\usepackage{enumitem}
\graphicspath{{pictures/}}
\DeclareGraphicsExtensions{.png,.jpg}

\renewcommand{\theenumi}{\cdot}

%%% Дополнительная работа с математикой
\usepackage{amsmath,amsfonts,amssymb,amsthm,mathtools} % AMS
\usepackage{icomma} % "Умная" запятая: $0,2$ --- число, $0, 2$ --- перечисление

%% Номера формул
\mathtoolsset{showonlyrefs=true} % Показывать номера только у тех формул, на которые есть \eqref{} в тексте.

%% Шрифты
\usepackage{euscript}	 % Шрифт Евклид
\usepackage{mathrsfs} % Красивый матшрифт


\usepackage{xcolor}
\usepackage{hyperref}

 % Цвета для гиперссылок
\definecolor{linkcolor}{HTML}{799B03} % цвет ссылок
\definecolor{urlcolor}{HTML}{799B03} % цвет гиперссылок
 
\hypersetup{pdfstartview=FitH,  linkcolor=linkcolor,urlcolor=urlcolor, colorlinks=true}

%% Свои команды
\DeclareMathOperator{\sgn}{\mathop{sgn}}

%% Перенос знаков в формулах (по Львовскому)
\newcommand*{\hm}[1]{#1\nobreak\discretionary{}
	{\hbox{$\mathsurround=0pt #1$}}{}}

\makeatletter
\def\@seccntformat#1{%
  \expandafter\ifx\csname c@#1\endcsname\c@section\else
  \csname the#1\endcsname\quad
  \fi}
\makeatother

%%% Начало документа
\begin{document}

\begin{titlepage}
\newpage

\begin{center}
\vspace{2cm}
МОСКОВСКИЙ ФИЗИКО-ТЕХНИЧЕСКИЙ ИНСТИТУТ \\*
(ГОСУДАРСТВЕННЫЙ УНИВЕРСИТЕТ) \\*
\hrulefill \\ \vspace{0.3cm}
ФАКУЛЬТЕТ ИННОВАЦИЙ И ВЫСОКИХ ТЕХНОЛОГИЙ
\end{center}

\vspace{8em}

\vspace{2.5em}
 
\begin{center}
\Large{ \textsc{\textbf{Задача многомерного размещения и её приложения}} }
\end{center}

\vspace{5cm}
 
\begin{flushright}
Работу выполнил \\ студент 599 группы \\ Куприянов А.А. \\
\vspace{1.5em}
Научный руководитель: \\
К.ф.-м.н. \\ Мусатов Д.В.\\
\end{flushright}
 
\vspace{\fill}

\begin{center}
Долгопрудный \\ 2016
\end{center}

\end{titlepage}


\section{Введение}
Данный отчет почти полностью основан на диссертации $[1]$, являющейся моей основной литературой в этом семестре. Азы теории были изучены в статьях $[2]$, $[4]$ и \href{https://courses.openedu.ru/courses/course-v1:mipt+GAMETH+fall_2016/info}{ курсом "Теория игр" на платформе openedu}.

\section{Постановка задачи в конечном случае.}
\textbf{ Формулировка задачи многомерного размещения: }
\textbf{Формулировка 1}

\par
Задан конечный набор точек $(x_1, \dots, x_n)$ координатного вещественного $d$ - мерного пространства:  $\forall i = 1, \dots, n$ имеем $x_i \in \mathbb{R}^{n}$. Объемлющее пространство $\mathbb{R}^{n}$ снабжено нормой $\vert \vert \cdot \vert \vert$, не обязательно евклидовой. Требуется открыть несколько пунктов, или мощностей $m_1, \dots ,  m_k \in  \mathbb{R}^{n}$, и прикрепить к ним все точки с помощью отображения прикрепления $h : \{ 1, \dots, n \} \rightarrow \{ 1, \dots, k \} $ с тем, чтобы минимизировать функционал

\[ kg + \sum_{i = 1}^{n} \vert \vert x_i - m_{h(i)} \vert \vert \]

Пространство, на котором ищется минимум, -- это пространство, типовым элементом которого является пара, состоящая из конечного подмножества $\{ m_1, \dots, m_k \}$ произвольной мощности $k$, а также одного из всех возможных отображений $h : \{1, \dots, n \} \rightarrow \{ 1, \dots, k \}$ 

Любой элемент фазового пространства задачи (пространства, на котором
осуществляется минимизация функционала) мы назовём, как обычно,
допустимым планом, или решением. 

\section{Переформулировка задачи многомерного размещения.}
Оказывается, что почти всегда можно восстановить по разбиению $\pi = \{ S_1, \dots, S_k \} \ : \ N = S_1 \sqcup \dots \sqcup S_k $ оптимальное решение, состоящее из набора $\{ m_1, \dots, m_k \}$ и отображения прикрепления $h^*$. Для этого нужно, чтобы члены коалиции $S$, образованной разбиением $\pi$, решили задачу поиска медианы для данной коалиции $S \subset N :$ \\
\[ \min \limits_{m \in \mathbb{R}^d} \left\{ \sum_{i \in S} \vert \vert x_i - m \vert \vert \right\} \]

Любое решение задачи поиска медианы для коалиции обозначим за $m[S]$ и назовем медианой группы $S$. Значение целевого функционала коалиции на любом решении обозначим за $D[S]$.

Для переформулировки введем величину средних общих издержек (монетарные + транспортные) для членов коалиции $S$, при условии выбора оптимальной локации для центра этой коалиции:

\[ c[S] = \min \limits_{m \in \mathbb{R}^d} \left\{ \dfrac{g + \sum \limits_{i \in S} \vert \vert x_i - m \vert \vert}{\vert S \vert} \right\} = \dfrac{g + D[S]}{\vert S \vert } \]

Тогда мы можем переформулировать ЗМР, описанную выше так: \\ \\ \\

\textbf{Формулировка 2}

\[ \min \limits_{k; \ \pi=\{ S_1, \dots, S_n \} \ : \ N = S_1 \sqcup \dotsc \sqcup S_k } \left \{ \sum_{l = 1}^{k} \vert S_l \vert c[S_l]  \right \} \]



Минимум теперь берётся просто по всем возможным разбиениям пространства
игроков $N$ на непересекающиеся коалиции (или группы), с заранее не
заданным количеством групп в разбиении.

\textbf{Лемма об эквивалентности формулировок ЗМР. }
\par
Формулировки 1 и 2 эквивалентны в следующем формальном смысле: значения целевых функционалов в точках оптимума совпадают и \par
$\Rightarrow$   Для любого решения ЗМР по формулировке 1, т.е. для пары $[(m_1, \dots, m_k); h(\cdot ) ]$ разбиением, полученным функцией прикрепления $h(\cdot)$,  достигается оптимум в формулировке 2.

$\Leftarrow $ Для любого выбора медиан $m_i \in M[S_i]$ внутри каждой коалиции $S_i$ разбиения  $\pi$, парой $[(m_1, \dots, m_k); h(\cdot)]$, где $h(\cdot)$ определяется разбиением $\pi$, достигается оптимум в формулировке 1.



\section{Лемма о медиане. По статье [1]}
 Если бы медиана любой коалиции была единственна, то разбиение на группы
однозначно определяло бы решение исходной задачи ЗМР, которая, в свою
очередь, всегда однозначно приводила бы к разбиению. Однако, увы, медиана
единственна далеко не всегда. \\
 
 Оказывается, что в евклидовом случае верен такой результат: \par
 \textbf{Лемма о медиане (Евклидов случай)} \par
 Рассмотрим пространство с обычной евклидовой нормой. Тогда для любой коалиции $S$, такой что все локации ее членов не лежат на одной прямой, решение $m[S] = M[S]$ единственное.
 \\
 
 \textbf{Замечание.} \par Действительно, если локации всех членов коалиции (точки $\{x_i\}_{i \in S}$) лежат на одной прямой, множеством медиан будет служить отрезок этой прямой, заключенный между двумя медианными локациями (после упорядочивания точек на прямой). 
 \\
 
 \textbf{Доказательство} (от противного) \par
 Предположим, что для некоторой локации $S$, удовлетворяющей условиям теоремы есть 2 решения задачи поиска медианы коалиции. Обозначим их за $m$ и $m'$. Положим $\bar{m} := \dfrac{m + m'}{2}$ - середина отрезка $mm'$. По условию леммы можно считать, что существует член коалиции $S$, локация которого не принадлежит отрезку $mm'$. \par
 Для $\forall i \in S \ : \ x_i \notin mm' $: отразим   $x_i$ относительно отрезка $mm'$. Получим точку $\tilde{x_i}$. Значит верно $\vert \vert x_i - m' \vert \vert = \vert \vert \tilde{x_i} - m \vert \vert$ и $\vert \vert x_i - \bar{m} \vert \vert = \dfrac{1}{2} \vert \vert \tilde{x_i} - x_i \vert \vert$. Запишем неравенство треугольника для $\triangle x_im\tilde{x_i}$ \[ \vert \vert x_i - m \vert \vert + \vert \vert \tilde{x_i} - m \vert \vert > \vert \vert \tilde{x_i} - x_i \vert \vert , \]  Откуда \[ \dfrac{\vert \vert x_i - m \vert \vert + \vert \vert x_i - m' \vert \vert}{2} > \vert \vert x_i - \bar{m} \vert \vert . \]
	
Суммируя полученное неравенство по всем резидентам (для резидентов, локации которых расположены на прямой, проходящей через $m$ и $m'$, неравенство треугольника выполняется в нестрогой форме) получаем,  что

\[ \sum_{i \in S} \vert \vert x_i - \bar{m} \vert \vert < \dfrac{1}{2} \left( \sum_{i \in S} \vert \vert x_i - m \vert \vert + \sum_{i \in S} \vert \vert x_i - m' \vert \vert \right) . \] 

Так как $m$ и $m'$ -- медианы коалиции, то
\[ D[S] = \sum_{i \in S} \vert \vert x_i - m \vert \vert = \sum_{i \in S} \vert \vert x_i - m' \vert \vert , \]

Откуда

\[ D[S] = \dfrac{1}{2} \left( \sum_{i \in S} \vert \vert x_i - m \vert \vert + \sum_{i \in S} \vert \vert x_i - m' \vert \vert \right) . \]

Но получили, что значение расстояний локаций резидентов до $\bar{m}$ меньше $D[S]$ -- противоречит выбору $m$ и $m'$, как медиан $S$.

\begin{center}
	\includegraphics[width=10cm]{1}
\end{center} 

\textbf{Что и требовалось доказать.}

\section{Чуточку обо всем.}

Еще хотелось бы рассказать об изученных мною уточнениях ЗМР, устойчивости решений ЗМР (статья $[1]$), топологических фактах из брошюры $[6]$ и связанных с ЗМР задачах, предложенных Д.В. Мусатовым. Но боюсь, что это перевалило бы за разумные рамки объема отчета, который мне нужно было сделать.

\section{ЗМР на букве Т.}

Напоследок, дальнейшей моей задачей является решение задачи ЗМР и равномерного распределения на букве Т. \\ План действий: \par
Сформулировать аналогичные определения и проверить корректность некоторых теорем статьи [1] на букве Т. \par Далее написать программу, осуществляющую некоторый перебор локаций игроков для поиска контрпримеров на различные уточнения задачи ЗМР. Первая, вдохновляясь статьей $[5]$, изучить равномерное распределение на букве Т.
$\mathbin{F}$

\section{Список литературы.}

\begin{enumerate}[label=()]


\item [ [  1] ] Савватеев А.В. Задача многомерного размещения и её приложения: теоретико-игровой подход. – Диссертация на соискание ученой степени
доктора физико-математических наук РЭШ, 2013
– 358 с. 
\item [ [  2] ] Ауманн Р., Шепли Л. Значения для неатомических игр. – М.: Мир, 1977.
– 358 с.
\item [ [  3] ] Захаров А.В. Теория игр в общественных науках. М.: препринт НИУ ВШЭ, 2014
\item [ [  4] ] Данилов В.И., Лекции по теории игр. РЭШ, 2002

\item [ [  5] ] Мусатов Д. Размер и число жителей регионов при однопиковой
плотности населения. – Магистерская работа РЭШ, 2008.

\item [ [  6] ] Данилов В.И., Лекции о неподвижных точках, РЭШ, 2006.
\end{enumerate}


\end{document}